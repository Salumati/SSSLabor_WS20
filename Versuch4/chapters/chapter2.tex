\chapter{Versuch 2}
\label{chap:VERSUCH-2}


\section{Fragestellung, Messprinzip, Aufbau, Messmittel}
\label{chap:VERSUCH-2-FRAGESTELLUNG}

\subsection*{Fragestellung}
Im 2. Teil des Versuchs soll ein Spracherkenner gebaut werden, der vier verschiedene Worte erkennt. Dafür wird jedes Wort 5 mal aufgenommen,
und ein Prototyp ausgerechnet indem von allen 5 die Spektren berechnet werden und anschließend der Mittelwert. Zum Vergleich der Prototypen mit einem
anderen Signal wird der Korrelationskoeffizient nach Bravais-Pearson berechnet.

\subsection*{Messprinzip, Aufbau und Messmittel}

Wie in Versuch 1 (add link!)


\section{Messwerte}
\label{chap:VERSUCH-2-MESSWERTE}

Im folgenden sind die gesprochenen Wörter hoch, tief, links und rechts dargestellt in der aufgezählten Reihenfolge:

\includegraphics[scale=1]{media/Sprachinput/Ref/hoch_Ref_0.png}
\captionof{figure}{Zeitspektrum-Ref-hoch}

\includegraphics[scale=1]{media/Sprachinput/Ref/tief_Ref_0.png}
\captionof{figure}{Zeitspektrum-Ref-tief}


\includegraphics[scale=1]{media/Sprachinput/Ref/links_Ref_0.png}
\captionof{figure}{Zeitspektrum-Ref-links}


\includegraphics[scale=1]{media/Sprachinput/Ref/rechts_Ref_0.png}
\captionof{figure}{Zeitspektrum-Ref-rechts}


\section{Auswertung}
\label{chap:VERSUCH-2-AUSWERTUNG}

Für den Spracherkenner brauchen wir zunächst die Referenzspektren für die jeden der vier Befehle ”Hoch”, ”Tief”, ”Links” und ”Rechts”, wobei jedes Wort 5 mal aufgesagt wurde und vom selber Sprecher stammen.
Dazu haben wir jeweils die Spektren der vier Wörter mit der Windowing Methode aus Aufgabe 1 berechnet und daraus das Referenzspektrum durch Mittelung aller 5 Spektren erhalten.
Im folgenden sind die Referenz Spekten dargestellt in der aufgezählten Reihenfolge.

\includegraphics[scale=1]{media/hoch_Windowing_Amplitudenspektrum_Mean.png}
\captionof{figure}{Amplitudenspektrum-Ref-hoch}


\includegraphics[scale=1]{media/tief_Windowing_Amplitudenspektrum_Mean.png}
\captionof{figure}{Amplitudenspektrum-Ref-tief}


\includegraphics[scale=1]{media/links_Windowing_Amplitudenspektrum_Mean.png}
\captionof{figure}{Amplitudenspektrum-Ref-links}


\includegraphics[scale=1]{media/rechts_Windowing_Amplitudenspektrum_Mean.png}
\captionof{figure}{Amplitudenspektrum-Ref-rechts}



Nachdem die Referenzspektren ermittelt wurden müssen sie mit Trainingssprektren verglichen werden, um zu erkennen wie gut die Worte erkannt werden.
Hierzu nehmen zwei Sprecher, der ursprüngliche Sprecher und ein weiterer, nochmal fünf Aufnahmen pro Wort auf.
Diese werden wieder mit der Windowing Methode in ihr Amplitudenspektrum zerlegt, allerdings wird dieses mal kein Mittelwert berechnet.

Um nun die Spektren vergleichen zu können benötigen wir den Korrelationskoeffizienten nach Bravais-Pearson, welcher uns ein Ähnlichkeitsmaß für die Spektren liefert.
Für den Korrelationskoeffizienten brauchen wir zuerst die Kovarianz:

\begin{equation}
	\sigma_fg = \frac{1}{n} \sum_{k =1}^n (f_k - \mu_f) * (g_k - \mu_g) 
\end{equation}
\begin{itemize}
	\item $\sigma_fg $ = Kovarianz
	\item f und g = die zu Vergleichenden Spektren
	\item $\mu$ = Mittelwert des jeweiligen Spektrums
\end{itemize}

Aus der Kovarianz lässt sich nun der Korrelationskoeffizient berechnen:

\begin{equation}
	r_fg = \frac{\sigma_fg}{\sigma_f * \sigma_g}
\end{equation}
\begin{itemize}
	\item $\sigma_f / \sigma_g$ = Standardabweichung vom jeweiligen Spektrum 
\end{itemize}

Wobei ein Koeffizient von 1 für ein identisches Spektrum steht, während ein Koeffizient von 0 für ein komplett unterschiedliches Spektrum steht.
Der Spracherkenner ermittelt den Koeffizienten für jedes Referenzspektrum und interpretiert das Eingangssignal als das Wort mit der höchsten Korrelation.

Um nun zu testen wie gut der Spracherkenner funktioniert werden alle Aufnahmen der beiden Testdatensätze vom Spracherkenner interpretiert und das Ergebnis überprüft.
Folgende Tabelle stellt die Erkennungsrate und Fehlerrate dar:


\begin{center}

\begin{figure}[h]

\begin{tabular}{|c|c|r|c|c|r|}
Wort & erkannt & Erkennungsrate & nicht erkannt & Fehlerrate &  stattdessen erkannt als: \\ 

\hline
Hoch(P1) & 5 & 100 \% & 0 & 0 \% & - \\ 
\hline
Tief (P1) & 3 & 60 \% & 2 & 40 \% & rechts \\ 
\hline
Links (P1) & 1 & 20 \% & 4 & 80 \% & hoch \\ 
\hline
Rechts (P1) & 4 & 80 \% & 1 & 20 \% & links \\ 
\hline
gesammt (P1) & 13 & 65 \% & 7 & 35 \% & - \\ 
\hline

Hoch (P2) & 3 & 60 \% & 2 & 40 \% & Links  \\  
\hline
Tief (P2) & 5 & 100 \% & 0 & 0 \% & - \\ 
\hline
Links (P2) & 0 & 0 \% & 5 & 100 \% & Tief \\ 
\hline
Rechts (P2) & 0 & 0 \% & 5 & 100 \% & Hoch \\ 
\hline
gesammt (P2) & 8 & 40 \% & 12 & 60 \% & - \\ 
\hline

\end{tabular} 
\label{Tab: Ergebnisse des SpracherkennerTests}
\caption{Tab: Ergebnisse des Spracherkenner Tests}

\end{figure}

\end{center}

P1 ist der Sprecher der auch das Referenzspektrum aufgenommen hat.


\section{Interpretation}
\label{chap:VERSUCH_2_INTERPRETATION}

Es zeigt sich das der erstellte Spracherkenner recht ungenau ist, selbst beim originalsprecher wurden nicht alle Befehle korrekt erkannt.
Grund hierfür könnte einerseits der geringe Datensatz für das Referenzspektrum sein, da es aus nur fünf Worten vom selben Sprecher erstellt wurde.
Auch die Mikrophonqualität könnte mit einspielen, vor allem da beide Sprecher mit unterschiedlichen Mikrophonen aufnehmen mussten, wodurch verschiedene Fehlerquellen aufeinander stoßen.
Zudem waren die Sprecher von unterschiedlichen geschlecht, wodurch ein größerer Unterschied in den Stimmlagen zu finden ist.

Für einen besseren Spracherkenner ist es sinnvoll das Referenzspektrum mit mehr Beispielaufnahmen, möglichst auch von unterschiedlichen Sprechern zu erstellen um einen genaueren Durchschnitt zu erhalten.
Es könnte auch sinnvoll sein mehrere Referenzsprektren pro Wort zu haben, eines für tiefere Stimme und eines für höhere Stimmen.

