\chapter{Versuch 1}
\label{chap:VERSUCH_1}


\section{Fragestellung, Messprinzip, Aufbau, Messmittel}
\label{chap:VERSUCH_1_FRAGESTELLUNG}

\subsection*{Fragestellung}
Das Experiment wird begonnen indem ein beliebiges Sprachsignal mit einem Mikrofon aufgenommen wird.
Mit Hilfe einer Triggerfunktion soll das Signal so zugeschnitten werden, dass das Sample mit dem gesagten Wort beginnt.
Im Anschluss wird dieses Signal dann in sein Spektrum zerlegt und dargestellt.
Als letztes soll noch die Windowing Methode implementiert werden um das Signal genauer zu analysieren.


\subsection*{Messprinzip}
Die Fourieranalyse ist ein essenzielle Methoden für die Signalverarbeitung. Sie erlaubt es die Grundfrequenz sowie die Obertöne einer Schwingung zu ermitteln.
Durch Windowing lassen sich fastperiodische Signale wie Sprache und Musik besser fouriertransformieren, da bei ihnen nur ein Teilausschnitt und nicht das gerammte Signal untersucht werden muss.

\subsection*{Aufbau}
Der Aufbau besteht lediglich aus einem Mikrofon welches mit einem Computer verwbunden wurde um über Python das Sprachsignal auf zu nehmen.


\subsection*{Messmittel}
\begin{itemize}
	\item Mikrofon
\end{itemize}

\section{Messwerte}
\label{chap:VERSUCH_1_MESSWERTE}
Unser belibig aufgenommenes Wort ist: "Naturschutzgebiet"

\begin{center}
\includegraphics[scale=1]{media/randomVoice/randomVoiceInput.png}
\captionof{figure}{Fig: Zeitspektrum von Naturschutzgebiet}

\end{center}


\section{Auswertung}
\label{chap:VERSUCH_1_AUSWERTUNG}
Das zuvor genannte beispiel Wort sieht folgendermaßen aus mit der Trigger Funktion:

\begin{center}
\includegraphics[scale=1]{media/randomVoice/randomVoiceInput_trigger.png}
\captionof{figure}{Fig: Ausschnitt des  Zeitspektrum von Naturschutzgebiet}

\end{center}


Es wird darauf geachtet das die Signal immer gleich lang sind um sie miteinander vergleichen zu können. Deshalb wird wenn das Sprachsignal < 1 Sekunde lang ist der rest mit 0 aufgefüllt.


Dann wurde das getriggerte Signal im Aplituden Spektrum dargestellt:

\begin{center}
\includegraphics[scale=1]{media/randomVoice/randomVoiceInput_trigger_Amplitudenspektrum.png}
\captionof{figure}{Fig: Amplitudenspektrum von Naturschutzgebiet}
\end{center}


Das unten stehende Schaubild zeigt das Spektrum welches mit der Windowing Methode transformiert wurde.
Dazu haben wir das Signal in Abschnitte mit einer Länge von 512 Samples, die sich jeweils zur
Hälfte überlappen, zerlegt. Jedes Fenster wurde mit einer Gaußschen
Fensterfunktion gewichtet, die so gewählt wird, dass die Fensterbreite 4 Standardabweichungen
entspricht. In jedem Fenster haben wir eine lokale Fouriertransformation durchgeführt und
dann die Fouriertransformierte über alle Fenster gemittelt.

\begin{center}
\includegraphics[scale=1]{media/randomVoice/randomVoiceInput_trigger_Windowing.png}
\captionof{figure}{Fig: Amplitudenspektrum von "Naturschutzgebiet" durch windowing}

\end{center}




\section{Interpretation}
\label{chap:VERSUCH_1_INTERPRETATION}
Zu dem Trigger schaubild gibt es nicht viel zu sagen. Es wurde lediglich das Signal auf die 1 Sekunde geschnitten ab dem Zeitpunkt wo der Schwellpunkt übertreten wurde.
Viel interessanter sind hier die beiden Spektren. Zum einen haben wir hier das einfache Signal welches FFT transformiert wurde
und zum anderen das Spektrum welches mit einem Gauß als Windowing Funktion bestimmt wurde.
Es ist gut zu erkennen dass das zweite Spektrum viel sauberer berechnet wurde. Die Grundfrequenz und ihre folgenden Harmonischen
sind weiterhin gleich aber diese sind deutlich besser zu erkennen.