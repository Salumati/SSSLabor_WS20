\chapter{Versuch 3}
\label{chap:VERSUCH_3}

\section{Fragestellung, Messprinzip, Aufbau, Messmittel}
\label{chap:VERSUCH_3_FRAGESTELLUNG}

\subsection{Fragestellung}
Im dritten Versuch geht es um die Ermittlung der Fehlerfortpflanzung. Hierzu werden die Seiten eines DinA4 Blattes gemessen und dann dessen Flächeninhalt ermittelt.

\subsection{Messprinzip, Aufbau, Messmittel}
\label{subSec:MAM}
Das Messprinzip, der Aufbau und die Messmittel sind wie in den obrigen Versuchen, nur das ein DinA4 Blatt als Basis für den Abstand verwendet wird.

\section{Messwerte}
\label{chap:VERSUCH_3_MESSWERTE}

\includegraphics[scale=.75]{media/paperPage.png}
\captionof{figure}{fig: von Hand gemessene Werte}
\label{fig:WerteDinA4}

\section{Auswertung}
\label{chap:VERSUCH_3_AUSWERTUNG}

\subsection{Teil a: Ermittlung des Messfehlers}
\label{subS:Versuch3a}
Zunächst wird der Fehler der Spannungsmessung wie in Versuch 1 über die Standardabweichung für die Messung der langen Seite ermittelt:
\begin{equation}
	s_{\bar{l_v}} = 0.0177
\end{equation}
\begin{itemize}
	\item $l_v$ = Spannung der Messung der langen Seite
\end{itemize}

Der Messfehler muss zusätzlich mit einen Korrekturfaktor ausgeglichen werden.
Da lediglich einmal gemessen wurde ist der verwendete Korrekturfaktor $t = 1,84$.
Daraus ergibt sich für die \textbf{gemessene Volt der langen Seite} ein Fehler von:

\begin{equation}
	l_v = 0,6758 \pm 1,84 * 0,0177 V = 0,6758 \pm 0,0326 V
\end{equation}

Dabei liegt der wahre Wert der Messung mit einer Wahrscheinlichkeit von 68,27 im Vertrauensbereich $[0.6432V, 0.7084V]$ und mit einer Wahrscheinlichkeit von 95,45\% im Bereich von $[0.4508V, 0.9008V]$.

Da die Entfernung indirekt aus der gemessenen Spannung berechnet wird muss für die Umrechnung die Fehlerfortpflanzung berechnet werden.
Der Messfehler für die Distanz ergibt sich aus der Formel für die Umrechnung
\begin{equation}
	\Delta l_d = f'(l_v) * \Delta l_v
\end{equation}
\begin{itemize}
\item $\Delta l_d$ ist der Messfehler für die Distanz
\item $\Delta l_v$ ist der Messfehler für die gemessene Spannung
\item $f'(l_v)$ ist die Ableitung der Übertragungsfunktion
\end{itemize}

die Übertragungsfunktion:
\begin{equation}
	f(l_v) = e^b * x^a
\end{equation}
\begin{equation}
	f'(l_v) = e^b * a * l_v ^{a-1}
\end{equation}
d.h. der konkrete Messfehler $\Delta l_d$ ist:
\begin{equation}
	\Delta l_d = e^5,17 * -1,69 * l_v ^ -2,69 * \Delta l_v = 2,71
\end{equation}
Unter Berücksichtigung des oben genannten Korrekturfaktor ergibt sich folgendes als Ergebnis für die \textbf{Länge der langen DinA 4 Seite}:
\begin{equation}
	l_d = 34.04  ± -2.77 cm
\end{equation}


\subsection{Teil b: Flächenmessung}
\label{subS:Versuch3b}

Formel zur Flächenberechnung:
\begin{equation}
	F = l * b = f(x)
\end{equation}
\begin{itemize}
	\item F = Fläche
	\item l = Länge lange Seite des DinA 4 Blattes
	\item b = Länge kurze Seite des DinA 4 Blattes
\end{itemize}
Wird nun die Fläche berechnet ergibt sich der Messfehler laut Gaußschem Fehlerfortpflanzungsgesetz aus den partiellen Ableitungen der einzelnen Messgrößen
\begin{equation}
	\Delta F = \sqrt{(\frac{\delta}{\delta l}f(x) * \Delta l)^2 + (\frac{\delta}{\delta b}f(b) * \Delta l)^2}
\end{equation}
mit:
\begin{equation}
	\frac{\delta}{\delta l}f(x) * \Delta l = b * \Delta l
\end{equation}
\begin{equation}
	\frac{\delta}{\delta b}f(b) * \Delta l = l * \Delta b
\end{equation}

Die Länge der langen Seite und ihr Fehler sind aus \ref{subS:Versuch3a} bekannt, nach der selben Methode wurde für die kurze Seite folgende Länge und Fehler ermittelt:

\begin{equation}
	Länge kurze Seite = 21,6 \pm 1.20 cm
\end{equation}

Damit ergibt sich ein Messfehler von:
\begin{equation}
	\Delta F = 74.103
\end{equation}

somit ist das Ergebnis der Flächenrechnung:
\begin{equation}
	F = 733.78 \pm 74.10 cm^2
\end{equation}


\section{Interpretation}
\label{chap:VERSUCH_3_INTERPRETATION}

Wie die Auswertung zeigt, ist der Messfehler unter anderem dadurch geprägt, wie oft unter gleichen Bedingungen gemessen wird. Dadurch muss bei wenigen Messungen ein größerer Vertrauensbereich genommen werden um dem eigentlichen Wert näher zu kommen.

So wird unter anderem der Fläche eines DIN A4(210mm x 297mm) Blattes ganz gut mal ein DIN A6(105mm x 148mm) Blatt mit dran gehängt aufgrund Messfehler.
Aus den gegebenen werten lässt sich daraus schließen, dass ein recht großer Messfehler eingetreten ist als die lange Seite gemessen wurde. Die gemessene Breite liegt ziemlich nah an der DINA 4 Norm mit einer berechneten breite von 215mm während die Länge weit davon entfernt ist mit 340mm.
